\section{Traceability}\label{sec:traceability}

There are billions of products being manufactured every day through complex supply chains that can extend to all parts of the world. However, there is very little information on how, when and where these products originated, manufactured and used during their life cycle \cite{horiuchirastreabilidade}.

Before reaching the end consumer, the goods go through an often wide network of retailers, distributors, carriers, warehousing facilities and suppliers who participate in the design, production, delivery and sales process of a product, but in many cases. These steps are a dimension invisible to the consumer \cite{provenance2015}.

In \cite{gryna1998juran, Opara2001} Traceability is defined as the ability to preserve the identity of the products and their origins, so that the collection, documentation and maintenance of information related to all processes in the production chain must be ensured. For a food product, traceability represents the ability to identify where and how it was grown, as well as the ability to track its post-harvest history and to identify the processes performed at each step in the production chain through records. Traceability is required primarily for \cite{horiuchirastreabilidade}:
\begin{itemize}
\item Improve credibility with customers and consumers.
\item Ensure that only quality materials and components are present in the final product.
\item Better allocate responsibilities.
\item Identify products that are distinct but may be confused.
\item Enable the return of defective or suspect products.
\item Find the causes of failures and take steps to repair them at the lowest possible cost.
\end{itemize}

In \cite{opara2003traceability} six important elements are to be considered for traceability:

\begin{itemize}
\item Product Traceability: Determines the location of a product at any stage of the production chain. to facilitate logistics, inventory management, product recall, and information disclosure to consumers and customers.
\item Process Traceability: Identifies the type and sequence of activities that affected a particular product. This includes any interactions between the product and physical / mechanical, chemical and environmental factors that result in the transformation of raw material into value added products.
\item Genetic Traceability: determines the genetic constitution of the product.
\item Input Traceability: Determines the type and source of input such as fertilizers and livestock.
\item Traceability of Diseases and Pests: Tracks the epidemiology of pests such as bacteria and viruses.
\item Measurement Traceability: determines the measurement instruments, in addition to specifying the environmental, geospatial and temporal factors that influence data quality.
\end{itemize}

A traceability system allows any user to track products by providing information about them (eg originality, components, or locations) during production and distribution. Suppliers and retailers typically require independent, government-certified traceability service providers to inspect products throughout the supply chain. Vendors and retailers request traceability services for different purposes.Suppliers want to receive certificates to showcase their products. Retailers want to verify the origin and quality of products \cite{lu2017adaptable}.

Supply chain visibility, or traceability, is one of the key challenges encountered in the business world, with most companies having little or no information about their own second- and third-tier suppliers. Transparency and end-to-end visibility of the supply chain can help shape product, raw material, test control, and end product flow, enabling better operations and risk analysis to ensure better chain productivity \cite{abeyratne2016blockchain}.

Folinas et al. \cite{folinas2006traceability} identified that the efficiency of a traceability system depends on the ability to track and trace each individual asset and logistics units, in a way that enables continuous monitoring from firstly processed until final clearance by the consumer.

Aung and Chang \cite{aung2014traceability} and Golan \cite{golan2004traceability} set three main traceability objectives, namely: (1) better supply chain management, (2) product differentiation and quality assurance, and (3) better identification of non-compliant products. An additional objective is to maintain assurance of traceability in accordance with applicable regulations and standards. A complete traceability system will include components that manage \cite{vargas2017trazabilidad}:

\begin{enumerate}
\item Identification, marking and assignment of traceable objects, parties and locations.
\item Automatic capture (by scanning or reading) of movements or events involving an object.
\item Record and share traceability data, internally or with parts of a supply chain, so that visibility of what has occurred can be achieved.
\end{enumerate}