\section{Blockchain}\label{sec:blockchain}
Recently, cryptocurrency has attracted extensive attention from both industry and the academy. Bitcoin, which is often called the first cryptocurrency, had a huge success with the capital market coming to \$ 10 billion in 2016 \cite{coindesk}. Blockchain is the central mechanism of the Bitcoin and was first proposed in 2008 and implemented in 2009 \cite{nakamoto2008bitcoin}. The blockchain can be considered as a public ledger, in which All committed transactions are stored in a block chain. This chain grows continuously when new blocks are attached to it \cite{zheng2016blockchain}.

While the system of financial institutions that serve as third parties reliable processors for processing payments work well for most still suffers from the shortcomings inherent in the model based on confidence. In addition, the cost of mediation increases transaction costs, which limits the practical minimum size of the transaction and eliminates the possibility of small occasional transactions. To solve these problems, \cite{nakamoto2008bitcoin} defined an electronic payment system called Bitcoin, based on cryptographic proof rather than reliable, allowing either party willing to transact directly with each other without the need to a reliable third party.

This revolution began with a new marginal economy on the Internet. Bitcoin emerges as an alternative currency issued and not backed by a central authority, but by automated consensus among networked users. Its true uniqueness, however, lay in the fact that it did not require that users trust each other. Through self-policing algorithmically, any malicious attempt to circumvent the system would be rejected. In a precise and technical definition, Bitcoin is a digital money. which is transacted via the Internet in a decentralized system without bail, using a ledger called blockchain. It's a new way of combining peer-to-peer file sharing rent with public key encryption \cite{swan2015blockchain}.

For \cite{swan2015blockchain}, besides the currency ( "Blockchain 1.0"), smart contracts ("2.0") demonstrate how the blockchain is in a position to become the fifth disruptive computing paradigm after mainframes, PCs, Internet and mobile/ social networks. Bitcoin is starting to become a digital currency, but technology blockchain behind it can be much more significant.

The rapid growth in blockchain technology adoption and the development of applications based on this technology have begun to revolutionize financial services industries. In addition to bitcoin, common applications of blockchain usage varies from proprietary networks used to process financial claims, insurance claims to platforms that can issue and trade equity and corporate bonds \cite{michael2018blockchain}.

Potential benefits of blockchain are more than just economic. They extend to the political, humanitarian, social and scientific domains. Its technological capacity is already being harnessed by specific groups to solve real world problems.

\subsection{Blockchain Properties}\label{sec:propriedades}

Blockchain technology has key features such as centralization, persistence, anonymity and auditability. Blockchain can function in a decentralized environment that is activated by the integration several key technologies such as cryptographic hash, digital signature (based on asymmetric encryption) and distributed consensus engine. With blockchain technology, a transaction may occur in a decentralized manner. As a result, blockchain can greatly save the cost and improve efficiency \cite{zheng2016blockchain}. The main properties of the blockchain are considered innovative and enable rapid adoption for technology \cite{greve2018blockchain}:

\begin{itemize}
\item Decentralization;
\item Availability and integrity;
\item Transparency and auditability;
\item Immutability and Irrefutability;
\item Privacy and Anonymity;
\item Disintermediation;
\item Cooperation and Incentives.
\end{itemize}