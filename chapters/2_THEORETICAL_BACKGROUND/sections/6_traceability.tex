\section{Traceability}\label{sec:traceability}
A traceability system allows you to track products by providing information about them (eg originality, components, or locations) during production and distribution. Suppliers and retailers typically require independent, government-certified traceability service providers to inspect products throughout the supply chain. Vendors and retailers request traceability services for different purposes.Suppliers want to receive certificates to showcase their products. Retailers want to verify the origin and quality of products \cite{lu2017adaptable}.

Traceability systems typically store information in standard databases controlled by service providers. This centralized data storage becomes a single point of failure and risks tampering.


Folinas et al. \cite{folinas2006traceability} identified that the efficiency of a traceability system depends on the ability to track and trace each individual asset and logistics units, in a way that enables continuous monitoring from firstly processed until final clearance by the consumer.

Aung and Chang \cite{aung2014traceability} and Golan \cite{golan2004traceability} set three main traceability objectives, namely: (1) better supply chain management, (2) product differentiation and quality assurance, and (3) better identification of non-compliant products. An additional objective is to maintain assurance of traceability in accordance with applicable regulations and standards. A complete traceability system will include components that manage \cite{vargas2017trazabilidad}:

\begin{enumerate}
\item Identification, marking and assignment of traceable objects, parties and locations.
\item Automatic capture (by scanning or reading) of movements or events involving an object.
\item Record and share traceability data, internally or with parts of a supply chain, so that visibility of what has occurred can be achieved.
\end{enumerate}