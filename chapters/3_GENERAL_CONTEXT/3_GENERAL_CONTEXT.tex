\xchapter{General Context}{} %sem preambulo
\acresetall

Traceability systems typically store information in standard databases controlled by service providers. This centralized data storage becomes a single point of failure and risks tampering. As consequence, these systems results in trust problem, such as fraud, corruption, tampering and falsifying information. Likewise, by being a single point of failure, centralized system is vulnerable to collapse \cite{tian2017supply}.

Nowadays, a new technology called the blockchain presents a whole new approach based on decentralization. Blockchain enables end-to-end traceability, bringing a common technology language to the supply chain, while allowing consumers to access the assets history of these products through a software application \cite{galvez2018future}.

\section{Supply Chain Management and blockchain}\label{sec:scm}

The complex web of relationships that provide materials manufacture the components, assemble or mix the parts and deliver the final product to market is known as the supply chain \cite{buurman2002supply}.

In order to solve some problems with Supply chain visibility and traceability, many internet of things technologies, such as RFID and wireless sensor network-based architectures and hardware, has been applied. However these technologies doesn't guarantee that the information shared by supply chain members in the traceability systems can be trusted. As a centralized organization, it can become a vulnerable target for bribery, and then the whole system can not be trusted anymore \cite{tian2017supply}.

Blockchain and distributed ledger technology underpinning cryptocurrencies such as Bitcoin, represent a new and innovative technological approach to realizing decentralized trustless systems. Indeed, the inherent properties of this digital technology provide fault-tolerance, immutability, transparency and full traceability of the stored transaction records, as well as coherent digital representations of physical assets and autonomous transaction executions \cite{caro2018blockchain}.

Blockchain enables end-to-end traceability, bringing a common technology language to the supply chain, while allowing consumers to access the assets history of these products through a web app. The need to track products across the complex supply chain from mineral prospecting and exploration to the end consumer is increasingly common: checking environmental impacts, or simply ensuring transparency for consumers \cite{galvez2018future}.

When applied to the supply chain, digital product information such as prospecting source details, batch numbers, mining and processing data, storage and transportation details are digitally connected to mineral items and their information is entered into the blockchain at each step of the process \cite{caro2018blockchain}.

Instead of storing data in an shadowy network system, blockchain allow all the goods' information to be stored in a shared and transparent system for all the members along the supply chain \cite{tian2017supply}. Monfared \cite{abeyratne2016blockchain} argued about the potential benefit of blockchain technology in manufacturing supply chain. They proposed that the inherited characteristics of the blockchain enhance trust through transparency and traceability within any transaction of data, goods, and financial resources. And it could offer an innovative platform for new decentralized and transparent transaction mechanism in industries and business.

There are many members among the supply chain, including suppliers, producers, manufacturers, distributors, retailers, consumers and certifiers. Each of these members can add, update and check the information about the product on the blockchain as long as they register as a user in the system. Each product has also a unique digital cryptographic identifier that connects the physical items to their virtual identity in the system. This virtual identity can be seen as the product information profile. Users in the system also have their digital profile, which contains the information about their introduction, location, certifications, and association with products \cite{tian2017supply}.

Supply chain members can register themselves in the system as a user through the register, which can provide credentials and a unique identity to the members. After registration, a public and private cryptographic key pair will be generated for each user. The public key can be used to identify the identity of the user within the system and the private key can be used to authenticate the user when interacting with the system. This enables each product can be digitally addressed by the users when being updated, added, or exchanged to the next user in the downstream position of the supply chain. Administrator members are responsible for provide specific roles for each member/user \cite{caro2018blockchain}.

All members of the business network agree with the information acquired in each transaction. Once consensus is reached, no permanent record can be changed. Each information provides critical data that can potentially reveal security issues with the product in question \cite{galvez2018future}.

Smart contract encodes the combination of services and other conditions defined in the contract. Therefore, the smart contract can automatically verify and apply these conditions. It also verifies all information required by regulation to enable automated verification of regulatory compliance \cite{lu2017adaptable}. 

Smart contracts running on a blockchain can be accessed and called by all participants. A smart contract by default has no owner. Once deployed, its author has no special privileges. Unauthorized users may accidentally trigger a function without permission. Therefore, smart contracts must have an internal permission to verify contract permissions.

The smart contract structural design has a big cost impact if the blockchain is public. The cost of contract implementation depends on its size, because the code is stored in the blockchain, which implies data storage fees proportional to the size of the contract. Therefore, a structural design with more lines of code costs more. A blockchain consortium does not have coin or token, so the monetary cost is not a problem. However, blockchain size is still a design concern because it grows with each transaction and each participant has a replica of the entire blockchain. In addition, a more structural design may affect performance as it may require more transactions \cite{lu2017adaptable}.

Because blockchain technology is still at an early stage of development, there is a general lack of standards for implementation. A blockchain must be universal and adaptable to specific situations \cite{valenta2017comparison}. In addition, the need to agree on a particular type of blockchain to be used puts the parties under pressure. This is a major disadvantage as blockchain technology is progressing rapidly, and predicting the best choice for the future is quite difficult \cite{galvez2018future}.

On the other hand, there are advantages of applying the blockchain concept to the mineral supply chain. One of the most important is: all stakeholders involved in the supply chain (Raw material / Producer, Manufacturer, Distributor, Wholesaler, Retailer) are motivated by the need to demonstrate to customers the superior quality of their methods and products \cite{lu2017adaptable}. 

In addition to serving the functions of a traceability system, a blockchain can be used as a marketing tool. Because block chains are fully transparent\cite{iansiti2017truth} and participants can control the assets in them \cite{liao2011food}, they can be used to enhance image and reputation of a company \cite{van2007essentials}, drive loyalty among existing customers \cite{pizzuti2015global} and attract new ones \cite{svensson2009transparency}. 

In fact, companies can easily distinguish themselves from competitors by emphasizing transparency and monitoring product flow along the chain. In addition, quickly identifying a source of contamination or loss can help protect a company's brand image \cite{mejia2010traceability} and alleviate the adverse impact of media criticism \cite{dabbene2011food}.

Blockchain simplifies this challenging task by providing one-to-many data integration and process orchestration among participants. In addition, it provides a lexicon and ontology to describe asset attributes across the supply chain. This in turn facilitates the establishment of a data structure that can be used by smart contracts to automate claims, certifications and market operations. There are three elements to explain why the food supply chain can benefit from the blockchain concept, namely: transparency, efficiency, safety and security \cite{galvez2018future}.

\subsection{Transparency}\label{sec:transparency}

The main goals of a blockchain are to facilitate information exchange, create a digital twin of information and its workflow, and validate the quality of assets as they move along the chain. These goals are achieved by allowing each participant to share claims, evidence and assessments of each other's claims about the product. The journey of mineral resources along the supply chain is captured in a blockchain object called a "mineral bundle". At the end of the journey, the package is the combination of all information provided by stakeholders over the life of the mineral item. This information can be used to establish the provenance, quality, sustainability and many other attributes of mineral assets \cite{martin2017technology}.

\subsection{Efficiency}\label{sec:efficiency}
Blockchain is an infrastructure that allows new transactions between players who do not yet know or trust each other. Smart contracts are instructions that interface with the blockchain protocol to automatically evaluate and possibly post transactions on the blockchain \cite{raskin2017law}. 

Similarly, smart libraries are specialized sets of blockchain-compatible functionality that can be used locally or privately or shared and licensed to other blockchain participants and agents. All participants meet at the blockchain, can evaluate the statements made and notify their account holders when matches are found in quality, time, quantity, etc. Buyers and sellers are matched by a shared but reliable need for data that can be combined and used by either party. So traceability doesn't have to wait for large company consortia to use patterns and / or semi-mandatory or concentrated business practices to access the information \cite{galvez2018future}.

\subsection{Safety and protection}\label{sec:Safety}
Blockchains can also be used to emit and manage the creation of unique cryptographic tokens \cite{nystrom1999pkcs}. Tokens can be made to represent the collateral value between two participants (for example, future production to be sent in a specific field lot). In fact, tokens do not have to take the form of value exchange for financial settlement of invoices and contracts. Instead, they represent a license to publish information that becomes uniquely valued in proportion to the needs of others on the blockchain. The strategy around issuing these encryption tokens, which need not be implemented in the initial system, is still being defined \cite{galvez2018future}.

\section{Supply chain and mineral resources}\label{sec:scmMineral}

To begin with, the starting point of a supply chain is the extraction of raw materials and how they are first processed (preprocessed) by suppliers for delivery in the next step. The next step is called manufacturing, where the conversion process for raw materials takes place. Following this, the constructed products are passed to the distributors who are responsible for allocating them to multiple different intermediaries, such as wholesalers and retailers. Distributors also maintain an active inventory of products, as previous products are connected to suppliers. Subsequently, wholesalers do not sell products directly to the public, but to other retailers, whereas retailers dispose products purchased to end users. Finally, Consumers are who buy or receive goods or services for personal needs or use and not for commercial resale or trade purposes \cite{litke2019blockchains}.

The manufacturer can validate crucial information about the natural resources they collected by reading and verifying all tags that the latter includes in its transactions and then proceeding to the proper execution of manufacturing step. New transactions with new information tags, such as manufacturer name, field experience and more, are sent after the internship has completed. Then the products are delivered to distributors. Distributors are able to sell products to wholesalers and retailers. This process is represented by blockchain transactions that display important data tags, such as merchant and customer address, exchange value, product raw material quality, and more \cite{sauer2018extending}. 

As the distributors sell products to intermediates (generally not end users), they can check valuable tag information about the progress route until that stage, for example the raw material origin, manufactures company popularity, distributor address and others. Retailers can audit product's natural resource quality, and get the appropriate feedback before selling it to the consumer. After that, when a distributor send the product to the wholesaler by submitting a corresponding transaction, the latter tags, such as manufacturer name, field experience and others, are submitted after the completion of acts in a similar way. Wholesaler can check transaction data and execute their selling to another wholesaler or retailer company by submitting a new transaction. The same applies to the retailer companies. Finally, end users obtain the final product with a submitted transaction and is able to track and verify all aspects from the beginning of its supply chain journey \cite{litke2019blockchains}. 

\subsection{Exploration}\label{sec:Exploration}
Exploration is the starting point of a supply chain. Phase which encompasses Mineral Prospecting and Geological Survey.

\subsubsection{Prospecting/ Surveying}\label{sec:Prospecting}
Prospecting means the search of mineral occurrences/surveying means the research on a mineralised area, previously defined during the prospecting phase to be applied as a Mineral Right from the Mining Authority.

\subsubsection{Mining right application}\label{sec:Mining}
Through handing over documents to the Mining Authority (in Brazil it is \textit{Agência Nacional de Mineração} [ANM]).

\subsubsection{Geological Survey}\label{sec:GeologicalSurvey}
Geological Survey is a set of activities carried out to determine the features on quality and quantity of ore found in a given mineral deposit.

\subsubsection{Mineral reserve definition}\label{sec:MineralReserve}
this step is related to the quantity of ore found during the Geological Survey.

\subsection{Mine Development and Design}\label{sec:MineDevelopment}
In this phase is made the preparation of a deposit to become a mine, opening accesses, setting up fences, building benches.

\subsection{Production}\label{sec:Production}
Production phase including ore extraction and mineral processing or transformation.

\subsubsection{Quarrying/ Blasting plan}\label{sec:Quarrying}
Quantity of stone extracted, same quantity of mineral reserves depleted, type of stone extracted.

\subsubsection{Processing}\label{sec:Processing}
Quantity of stone crushed/processed, same quantity of stone in stock deducted, type of stone product, market requirements.

\subsubsection{Aggregate Product}\label{sec:AggregateProduct}
Listing of products with identified quantities and respective technological features.

\subsection{Distribution}\label{sec:Distribution}
Product distribution with identification of clients and their segmentation. 

\subsubsection{Client by Product Type}\label{sec:Client}
Name, address, size, sector, business registration numbers (federal, state, municipal, county).

\subsubsection{Transport}\label{sec:Transport}
Invoice, bills, type of transportation, weight, distance.

\subsection{Mine Closure}\label{sec:MineClosure}
Mine Closure or mineral decommissioning, it means the end of the mining activity and the start of the installations removal and area remodelling through plantation and/or building a lake.

\subsubsection{Legal Compliance/Decommissioning Plan}\label{sec:LegalCompliance}
Status of all documentation related to the Decommissioning Plan and its Legal Compliance.

\subsubsection{Environmental Recovery/ ANM Application}\label{sec:EnvironmentalRecovery}
Application number and status.

