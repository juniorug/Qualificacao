\section{Scientific Method}\label{sec:method}

For project development, the agile Scrum method has been used. In Scrum, projects are divided into cycles called sprints, with frequent meetings where the team can inform what is being done and think of ways to quickly improve the process. Scrum proposes constant project monitoring. Often the team will be meeting, exchanging experiences, evaluating what has been done and re-planning what will be done next.

During the requirements gathering, developers and other stakeholders sought to raise and prioritize the needs of future software users (referred as requirements). After the requirements gathering, in the requirements specification stage, developers made a detailed study of data collected in the previous activity, from where models were built to represent the software system being developed.

At the architectural design stage of the system two basic activities were performed: architectural design (or high level design), and detailed design (or low level design). Some aspects were considered at this stage of system design, such as: system architecture, platform used, Database Manager System (DBMS) used and graphical interface standard.

In the application development period, the backend and frontend components will be created from the computational description of the design phase. Pre-existing software tools and class libraries will be used to streamline the activity. These tools and libraries will be defined during the architectural design.

For system validation, two main requirements will be evaluated: the components and the behavior of who will use the application. For the first point, functional, integration and security tests will be performed. For the second, the \acf{TAM} method will be used to evaluate user acceptance, utility and ease of use.

The appendix \ref{app:projectManagement} presents the activities for project management, the user stories and non-functional requirements.