\documentclass[en, qual, classic, a4paper]{ifbathesis}

%% Preambulo:
\usepackage[utf8]{inputenc}
\usepackage[english]{babel}
\usepackage{graphicx}
\usepackage{lipsum}
\usepackage{hyphenat}
% \usepackage[usenames, dvipsnames, table]{xcolor}
\usepackage{float}
\usepackage{booktabs}
\usepackage{pifont}
\usepackage{multirow}
\usepackage{listings} 
\usepackage{colortbl}
\usepackage{xfrac}
\usepackage[FIGTOPCAP]{subfigure}
\usepackage[printonlyused, withpage]{acronym}
\usepackage{graphicx,url}
%\usepackage[hidelinks,breaklinks]{hyperref}
%\PassOptionsToPackage{hyphens}{url}

\newcommand{\subsubsubsection}[1]{\paragraph{#1}\mbox{}\\}
\setcounter{secnumdepth}{4}
\setcounter{tocdepth}{4}


\university{Instituto Federal da Bahia}
\address{Salvador}
\institute{Departamento de Pós-Graduação e Qualificação}
\library{Biblioteca Professor Raul Varella Seixas}
\program{Programa de P\'{o}s-Gradua\c{c}\~{a}o em Engenharia de Sistemas e Produtos}
\majorfield{Engenharia de Sistemas e Produtos}
\title{A blockchain framework for traceability in supply chain management}

% Data da defesa
% e.g. \date{19 de fevereiro de 2013}
\date{30 de outubro de 2019}
% e.g. \defenseyear{2013}
\defenseyear{2019}

% Autor
\author{Edivaldo Mascarenhas Ferreira de Jesus Júnior}

% Orientador(a)
\adviser{Manoel Carvalho Marques Neto}
\coadviser{Allan Edgard Silva Freitas}

%% Inicio do documento
\begin{document}
\ppgespfrontpage
\frontmatter
\ppgesppresentationpage


\input{sections/abstract}
\tableofcontents
\listoffigures
\listoftables
\chapter*{Lista de Siglas}
\begin{acronym}[PPGESP]
    \acro{PPGESP}{Programa de Pós-Graduação em Engenharia de Sistemas e Produtos}
    \acro{CNPq}{Conselho Nacional de Desenvolvimento Científico e Tecnológico}
    \acro{BG}{Byzantine generals}
    \acro{PoW}{proof-of-work}
    \acro{BFT}{Byzantine fault tolerance}
    \acro{SC}{smart contract}
    \acro{SCM-BP}{Supply Chain Management - Blockchain Platform}
    \acro{TAM}{Technology Acceptance Model}
\end{acronym}

%% Parte textual
\mainmatter

%Seções
\input{chapters/1_INTRODUCTION/1_INTRODUCTION.tex}
\xchapter{Theoretical Background}{In this section the main concepts studied are presented, which provided subsidies for the development of the proposed project.}


\input{chapters/2_THEORETICAL_BACKGROUND/sections/1_generalContext.tex}
\input{chapters/2_THEORETICAL_BACKGROUND/sections/2_blockchain.tex}
\input{chapters/2_THEORETICAL_BACKGROUND/sections/3_fundamentalsOfBlockchain.tex}
\section{Public Blockchain Versus Private Blockchain}\label{sec:versus}

Blockchain networks can be categorized into two groups: public (or permissionless) blockchain, and private (or federated or permissioned) blockchain (with permission and controlled access) \cite{greve2018blockchain}.

\subsection{Public blockchain}\label{sec:blockchainPublica}
On a public or permissionless blockchain, any person can participate without a specific identity. Public blockchains typically involve a native cryptocurrency and often use \acf{PoW} consensus and economic incentives \cite{androulaki2018hyperledger}. Public blockchain can be audited by anyone, and each node has as much transmission power as any other. For a transaction to be considered valid, it must be authorized by all nodes constituents via the consensus process. As long as each node meets protocol-specific stipulations, their transactions can be validated and thus added to the chain \cite{Comstor2018}.

As the P2P network node set is unknown, its membership is dynamic, allowing random node entrances and exits and also anonymity of them. Blockchain can act in global scale, without the control of its participants, who do not even trust each other mutually. Are examples of public blockchain the Bitcoin network, the Ethereum and several other cryptocurrencies \cite{bashir2018mastering, antonopoulos2017mastering}.

\subsubsection{Consensus for Public Blockchain}\label{sec:consensoPublica}
Due to the uncertainties regarding the participants, public blockchains generally adopt mining-based consensus mechanisms. In these mechanisms miners vie with each other for consensus leadership, using computational power, possession power over cryptocurrency or other election-relevant powers that cannot be monopolized such that the same knots always come out victorious) \cite{greve2018blockchain}.

Compensation to these miners for their work is often cryptocurrencies. These incentives are critical to prevent Byzantine attacks by solving the fundamental challenge of agreement on a global scale. Currently proof of work is one of the few successful and resilient consensus approaches to Sybil attacks \cite{douceur2002sybil} (impersonation attacks, when malicious users become impersonate others).

\subsection{Private Blockchain}\label{sec:blockchainPrivada}
Permissioned, federated or private blockchains, on the other hand, perform a blockchain between a set of known and identified participants. A private blockchain provides a way to protect the interactions between a group of entities that have a common goal but that don't totally trust each other, like companies that trade funds, assets or information. Relying on peer identities, one private blockchain may use the traditional consensus of \acf{BFT} \cite{androulaki2018hyperledger}.

Federated blockchain has its known composition. It is formed by $n$ processes whose inputs and outputs are subject to permissions. Participants are identified, authenticated and authorized. This model of blockchain aims to better serve corporate or private interests where participants have well-defined roles and can even organize themselves into groups. Examples of private blockchain are Hyperledger Fabric \cite{cachin2016architecture} and some other projects \cite{cachin2017blockchain}.

\subsubsection{Consensus for Private Blockchain}\label{sec:consensoPrivada}
Due to the fact that it is a controlled network with $n$ known participants and identified by the federation, classic \acf{BFT} protocols and deterministic Byzantine consensus can be adapted to the blockchain \cite{androulaki2018hyperledger}.

In addition, there is no need to use incentives to agreement, as the federation of stakeholders can establish its own financial model of remuneration. Incentives, however, may be used for other purposes but, different from evidence-based consensus, they are not essential to consensus \cite{greve2018blockchain}.

In the \ac{BFT} literature, replication consistency is maintained by two principles:

\begin{itemize}
\item No mistake: Leaders are prevented from making mistakes, so there is only one possible proposal per leader per rating.
\item Proposal Security: A (higher-ranked) proposal can extend, but not modify, any lower-ranking compromised log prefix \cite{abraham2017blockchain}.
\end{itemize}
\input{chapters/2_THEORETICAL_BACKGROUND/sections/5_smartContracts.tex}
\section{Traceability}\label{sec:traceability}

There are billions of products being manufactured every day through complex supply chains that can extend to all parts of the world. However, there is very little information on how, when and where these products originated, manufactured and used during their life cycle \cite{horiuchirastreabilidade}.

Before reaching the end consumer, the goods go through an often wide network of retailers, distributors, carriers, warehousing facilities and suppliers who participate in the design, production, delivery and sales process of a product, but in many cases. These steps are a dimension invisible to the consumer \cite{provenance2015}.

In \cite{gryna1998juran, Opara2001} Traceability is defined as the ability to preserve the identity of the products and their origins, so that the collection, documentation and maintenance of information related to all processes in the production chain must be ensured. For a food product, traceability represents the ability to identify where and how it was grown, as well as the ability to track its post-harvest history and to identify the processes performed at each step in the production chain through records. Traceability is required primarily for:
\begin{itemize}
\item Improve credibility with customers and consumers.
\item Ensure that only quality materials and components are present in the final product.
\item Better allocate responsibilities.
Produtos Identify products that are distinct but may be confused.
\item Enable the return of defective or suspect products.
\item Find the causes of failures and take steps to repair them at the lowest possible cost.
\end{itemize}

In \cite{opara2003traceability} six important elements are to be considered for traceability:

\begin{itemize}
\item Product Traceability: Determines the location of a product at any stage of the production chain. to facilitate logistics, inventory management, product recall, and information disclosure to consumers and customers.
\item Process Traceability: Identifies the type and sequence of activities that affected a particular product. This includes any interactions between the product and physical / mechanical, chemical and environmental factors that result in the transformation of raw material into value added products.
\item Genetic Traceability: determines the genetic constitution of the product.
\item Input Traceability: Determines the type and source of input such as fertilizers and livestock.
\item Traceability of Diseases and Pests: Tracks the epidemiology of pests such as bacteria and viruses.
\item Measurement Traceability: determines the measurement instruments, in addition to specifying the environmental, geospatial and temporal factors that influence data quality.
\end{itemize}

A traceability system allows any user to track products by providing information about them (eg originality, components, or locations) during production and distribution. Suppliers and retailers typically require independent, government-certified traceability service providers to inspect products throughout the supply chain. Vendors and retailers request traceability services for different purposes.Suppliers want to receive certificates to showcase their products. Retailers want to verify the origin and quality of products \cite{lu2017adaptable}.

Supply chain visibility, or traceability, is one of the key challenges encountered in the business world, with most companies having little or no information about their own second- and third-tier suppliers. Transparency and end-to-end visibility of the supply chain can help shape product, raw material, test control, and end product flow, enabling better operations and risk analysis to ensure better chain productivity \cite{abeyratne2016blockchain}.

Traceability systems typically store information in standard databases controlled by service providers. This centralized data storage becomes a single point of failure and risks tampering.


Folinas et al. \cite{folinas2006traceability} identified that the efficiency of a traceability system depends on the ability to track and trace each individual asset and logistics units, in a way that enables continuous monitoring from firstly processed until final clearance by the consumer.

Aung and Chang \cite{aung2014traceability} and Golan \cite{golan2004traceability} set three main traceability objectives, namely: (1) better supply chain management, (2) product differentiation and quality assurance, and (3) better identification of non-compliant products. An additional objective is to maintain assurance of traceability in accordance with applicable regulations and standards. A complete traceability system will include components that manage \cite{vargas2017trazabilidad}:

\begin{enumerate}
\item Identification, marking and assignment of traceable objects, parties and locations.
\item Automatic capture (by scanning or reading) of movements or events involving an object.
\item Record and share traceability data, internally or with parts of a supply chain, so that visibility of what has occurred can be achieved.
\end{enumerate}
\input{chapters/2_THEORETICAL_BACKGROUND/sections/7_supplyChainBlockchain.tex}
\xchapter{Technical Specification}{} %sem preambulo

\acresetall 

\ac{SCM-BP} is a generic framework intended to be used in any kind of supply chain correlated to assets and products. 
Craton-Roche SCM-BP is a use case of this framework applied, initially, to the mining supply chain and more specifically, to the gravel ecosystem.

\section{Application Architecture}\label{sec:applicationArchitecture}

\ac{SCM-BP} is divided into three main modules described below: WebApp - FrondEnd, WebApp - BackEnd and Data Storage. Figure~\ref{fig:detalhamentotecnico} shows the application architecture and its components. Figure~\ref{fig:dataStructure} present the main data structure.

\begin{figure}[htbp]
\begin{center}
  \includegraphics[scale=0.55]{images/detalhamentotecnico.png}
\caption{Application architecture of \ac{SCM-BP}}
\label{fig:detalhamentotecnico}
\end{center}
\end{figure}

\subsection{WebApp - FrondEnd}\label{sec:WebAppFrondEnd}
WebApp - FrontEnd is a client–server computer application which the client (including the user interface and client-side logic) runs in a web browser. This is a single-page application (SPA), a web application that interacts with the user by dynamically rewriting the current page rather than loading entire new pages from a server. This approach avoids interruption of the user experience between successive pages, making the application behave more like a desktop application.

The Application is build with React (also known as React.js or ReactJS). This is a JavaScript library for building user interfaces. It is maintained by Facebook and a community of individual developers and companies. Used as a base in the development of single-page or mobile applications, React is optimal for fetching rapidly changing data that needs to be recorded. However, fetching data is only the beginning of what happens on a web page, which is why complex React applications usually require the use of additional libraries for state management, routing, and interaction with an API.

The Webapp - FrontEnd is divided into two main blocks and these are classified according to the interactions: User Interaction Modules and Backend Interactions Services.

\subsubsection{User Interaction}\label{sec:UserInteraction}
The User Interaction modules are responsible for providing web pages that will be rendered on client’s web browser. These interactions are provided by web pages grouped by the following modules:

\begin{itemize}
\item Login page
\item Application configuration module
\item User handling module (actors - CRUD)
\item Data entry module (forms)
\item Data visualization module
\item Reporting module
\end{itemize}

\subsubsubsection{Login Module}
The Login Module is responsible for display the login and authentication alternatives pages (‘forgot my password’, ‘reset my password’, etc.).

\subsubsubsection{Application configuration module}
The Application configuration module provides the features of creation/configuration of supply chain items and supply chain flows (steps and sub-tasks).

\subsubsubsection{User handling module}
This module provides the features for creation/configuration of Actors and Roles. The table in appendix \ref{app:userCreationFields} show the fields and values for creating a user.

\subsubsubsection{Data entry module}
The Data entry module provides form pages that allow users to enter data in the application, search and move assets from a step to another.

\subsubsubsection{Data visualization module}
The Data visualization module is responsible to display the information about assets in the supply chain flow. 

\subsubsubsection{Reporting Module}
In the Reporting module users can generate reports/files containing information organized in a narrative, graphic, or tabular form, prepared on ad hoc, periodic, recurring, regular, or as required basis. Reports may refer to specific periods, events, occurrences, or subjects, and may be presented in written form or any other format.

\subsubsection{Backend Interaction}\label{sec:BackendInteraction}
Backend interactions happen via a service layer consisting of:

\begin{itemize}
\item Authentication service
\item Application setup service
\item User creation service (actors)
\item Data entry service (forms)
\item Data visualization service
\item Reporting service
\end{itemize}

\subsubsubsection{Authentication Service}
The function of the Authentication Service is to request information from an authenticating party, and validate it against the configured identity repository using the specified authentication module. After successful authentication, the user session is activated and can be validated across all web applications participating in an SSO environment. For example, when a user or application attempts to access a protected resource, credentials are requested by one (or more) authentication modules. Gaining access to the resource requires that the user or application be allowed based on the submitted credentials.

\subsubsubsection{Application setup Service}
Application setup service provides methods to configure and edit  supply chain items and supply chain flows, defining which steps and sub-tasks will be present in this flow and which information will be present in these steps.

\subsubsubsection{User creation Service}
This service is responsible for the creation of users and roles, to allow them to log in and use the application’s features. Only Administrators are allowed to create new users (see Actions and Actors).

\subsubsubsection{Data entry Service}
Data entry service receives data from UI forms and send them to the backend to be processed and stored.

\subsubsubsection{Data visualization Service}
Data visualization services provides information about the supply chain: Assets, users and transactions, to be used by the data visualization module.

\subsubsubsection{Reporting Service}
Report services generate files (Doc/PDF/XSL, etc...) from a specific period of time with information about the supply chain: Assets, users and transactions.
\subsection{WebApp - BackEnd}\label{sec:WebAppBackEnd}
WebApp - BackEnd is a Middleware that runs on the server. This Middleware (server-side software) facilitates client-server connectivity, forming a middle layer between the app(s) and the network: the server, the database, the operating system, and more. It receives requests from the clients (in this case, the WebApp - FrontEnd), and contains the logic to send the appropriate data back to the applicant, over HTTP and REST.  These are the main conventions that provide structure to the request-response cycle between clients and servers.

WebApp - BackEnd is an application build with Node.js, an application platform where developers can write Javascript programs that are compiled, optimized and interpreted by the V8 virtual machine. Node.js can create quick, reliable websites and products in much efficient manner. Developing easy to scale real time applications in other technologies is bit difficult, but JavaScript technologies made it easier.

The WebApp - BackEnd is composed by the API Gateway, Service Layer and Resource Locator more detailed below.

\subsubsection{API Gateway}\label{sec:APIGateway}
API Gateway is a managed service that enables easily create, publish, maintain, monitor and secure REST APIs to act as a "gateway" for applications to access data, business logic, or functionality in the backend services, such as workloads. The API Gateway provides a simple uniform view of external resources to the internals of an application. It manages all tasks involved in receiving and processing API calls, including traffic management, authorization and access control, monitoring and management of API versions.

\begin{figure}[htbp]
\begin{center}
  \includegraphics[scale=0.75]{images/apigateway.png}
\caption{API Gateway.}
\label{default-regular2}
\end{center}
\end{figure}

An API is a collection of clearly defined methods of communication between different software components. More specifically, a Web API is the interface created by the back-end: the collection of endpoints and the resources these endpoints expose. A Web API is defined by the types of requests that it can handle, which is determined by the routes that it defines, and the types of responses that the clients can expect to receive after hitting those routes. One Web API can be used to provide data for different front-ends. Since a Web API can provide data without really specifying how the data is viewed, multiple different HTML pages or mobile applications can be created to view the data from the Web API.

Basically, the Gateway is an interface that receives calls to its internal systems, being a large gateway. 

It can act in five different ways:

\begin{itemize}
\item Filter for call traffic from different media (web, mobile, cloud, among others);
\item Single gateway to the various APIs you want to expose;
\item Essential component of API management, as API Suite;
\item Router: API and Rate Limit traffic router;
\item Security engine with authentication, logging and more.
\end{itemize}

Gateway access can be done from many different devices. Therefore, it must have the power to unify outgoing calls and be able to deliver to the user content that can be accessed from any browser and system. The 6 Benefits of a Gateway API:

\begin{enumerate}
\item Application Layer Separation and different requests:
One of the best benefits of this layer is that a gateway can clearly separate implemented APIs and microservices from the people who will actually use them.
\item Increased simplicity for the consumer: Using a Gateway, is possible to show to end user a unique front end with an API collection, and can be much more transparent with API users.
\item Development Improvement: Separation of purposes and functionality not only makes development focus much more on what is really needed, but also helps the server withstand the information demand for the services used. For example, a service that is called a few times a day needs fewer resources than a service called all the time, making the most of the machine's performance.

\item Buffer Zone against attacks: By utilizing multiple standalone Gateway-controlled services, any attack on the application will not affect the overall system, just that service, keeping everything running smoothly. This is the Buffer Zone. In addition to security, this strategy makes it much simpler for the user, as all other features remain normal, not causing stress.

\item Dedication of Services in Favor of User Experience: With the API service independence strategy, a developer can have all the documentation needed to use in a much simpler way, optimizing their time and dedicating themselves exclusively to their activity. This way is possible to usage SDKs for each API separately to make documentation as specific as possible.

\item Activity log anticipating errors: Since all calls to services will go through the Gateway, controlling them all is very simple. This type of log can give the owner of the API a very high power. With it, is possible to find all the errors that can bring down any service, and even who is responsible for a good consumption of the API. This way is easier to predict the number of possible calls avoiding any problems for users.

\end{enumerate}

Gateways as a Security Feature: In the APIs world, one of the most subject talked about issues is always security, and having an API Gateway is one of the best solutions on the market to get full control of API’s, because this pattern addresses the so-called CIA (Confidentiality, Integrity, Availability) almost flawlessly.

\subsubsection{Service Layer}\label{sec:ServiceLayer}
A Service Layer defines an application's boundary [Cockburn PloP] and its set of available operations from the perspective of interfacing client layers. It encapsulates the application's business logic, controlling transactions and coordinating responses in the implementation of its operations.

Enterprise applications typically require different kinds of interfaces to the data they store and the logic they implement: data loaders, user interfaces, integration gateways, and others. Despite their different purposes, these interfaces often need common interactions with the application to access and manipulate its data and invoke its business logic. The interactions may be complex, involving transactions across multiple resources and the coordination of several responses to an action. Encoding the logic of the interactions separately in each interface causes a lot of duplication.

Using service layer pattern provides some benefits:

\begin{enumerate}
\item Centralizes external access to data and functions.
\item Hides (abstracts) internal implementation and changes.
\item Allows for versioning of the services.
\end{enumerate}

The service layer acts as an orchestrator, controlling the flow of incoming and outcoming information requests and responses. Orchestration allows to directly link process logic to service interaction within workflow logic. This combines business process modeling with service-oriented modeling and design, realizing workflow management through a process service model. Orchestration brings the business process into the service layer, positioning it as a master composition controller.

\subsubsection{Resource Locator}\label{sec:ResourceLocator}

Resource locators are components that abstracts the persistence layer. Their job is to provide an object that can help services to discover and persist information from/to the Data Storage Module. Information can be stored in the Blockchain, Filesystem or Database and resource locators should know exactly where get/put data within them.  
\subsection{Data Storage}\label{sec:DataStorage}
Data storage is a general term for archiving data in electromagnetic or other forms for use by a computer or device. Different types of data storage play different roles in a computing environment. In addition to forms of hard data storage, there are now new options for remote data storage, such as cloud computing, and blockchain that can revolutionize the ways that users save and access data.  

SCM-BP uses three applications as data storages: Blockchain, Cloud filesystem and relational database better detailed on next subsections. Blockchains grow continuously because of the amount of data and code in them, which is unchanging. Therefore, an important design decision is to choose which data and calculations to keep in and out of the chain.

\subsubsection{Blockchain}\label{sec:DataStorageBlockchain}
A blockchain is a peer-to-peer distributed ledger forged by consensus, combined with a system for “smart contracts” and other assistive technologies. Together these can be used to build a new generation of transactional applications that establishes trust, accountability and transparency at their core, while streamlining business processes and legal constraints.

SCM-BP uses Blockchain as a supply chain that track parts and service provenance, ensure authenticity of goods, block counterfeits and reduce conflicts.

To achieve that, Hyperledger Fabric is used. Hyperledger is an open source collaborative effort created to advance cross-industry blockchain technologies. It is a global collaboration, hosted by The Linux Foundation, including leaders in finance, banking, Internet of Things, supply chains, manufacturing and Technology.
Hyperledger Fabric is an enterprise-grade permissioned distributed ledger framework for developing solutions and applications. Its modular and versatile design satisfies a broad range of industry use cases. It offers a unique approach to consensus that enables performance at scale while preserving privacy.

In context of SCM-BP, the Blockchain module consists in a smart contract, chaincode and the ledger. From the application developer’s perspective, a smart contract, together with the ledger, form the heart of a Hyperledger Fabric blockchain system. Whereas a ledger holds facts about the current and historical state of a set of business objects, a smart contract defines the executable logic that generates new facts that are added to the ledger. A chaincode is typically used by administrators to group related smart contracts for deployment, but can also be used for low level system programming of Fabric.

\subsubsubsection{Smart contract}
Before businesses can transact with each other, they must define a common set of contracts covering common terms, data, rules, concept definitions, and processes. Taken together, these contracts lay out the business model that govern all of the interactions between transacting parties.

A smart contract defines the rules between different organizations in executable code. Applications invoke a smart contract to generate transactions that are recorded on the ledger.

\subsubsubsection{Chaincode}
Hyperledger Fabric users often use the terms smart contract and chaincode interchangeably. In general, a smart contract defines the transaction logic that controls the lifecycle of a business object contained in the world state. It is then packaged into a chaincode which is then deployed to a blockchain network. Think of smart contracts as governing transactions, whereas chaincode governs how smart contracts are packaged for deployment.

\subsubsubsection{Ledger}
At the simplest level, a blockchain immutably records transactions which update states in a ledger. A smart contract programmatically accesses two distinct pieces of the ledger – a blockchain, which immutably records the history of all transactions, and a world state that holds a cache of the current value of these states, as it’s the current value of an object that is usually required.

Smart contracts primarily put, get and delete states in the world state, and can also query the immutable blockchain record of transactions.

\begin{itemize}
\item A \textbf{get} typically represents a query to retrieve information about the current state of a business object.
\item A \textbf{put} typically creates a new business object or modifies an existing one in the ledger world state.
\item A \textbf{delete} typically represents the removal of a business object from the current state of the ledger, but not its history.
\end{itemize}

Smart contracts have many APIs available to them. Critically, in all cases, whether transactions create, read, update or delete business objects in the world state, the blockchain contains an immutable record of these changes.

\subsubsection{Filesystem}\label{sec:Filesystem}
A cloud file system is a tiered storage system that provides shared access to file data. Users can create, delete, modify, read and write files, as well as logically organize them into directory trees for intuitive access.

Cloud file sharing can be defined as a service that gives multiple users simultaneous access to a cloud file data set. Cloud file sharing security is managed with user and group permissions, allowing administrators to tightly control access to shared file data.

For all file uploaded and stored in the filesystem, a locally stored digital fingerprint (hash) is saved in the blockchain, separately from the original files or content, to make it easier to confirm whether data has been altered or manipulated in a particular organization.

\subsubsection{Database}\label{sec:Database}
A relational database is a set of formally described tables from which data can be accessed or reassembled in many different ways without having to reorganize the database tables. The standard user and application programming interface (API) of a relational database is the Structured Query Language (SQL). SQL statements are used both for interactive queries for information from a relational database and for gathering data for reports.

\section{Actions And Actors}\label{sec:actionsAndActors}

The system is governed by a set of rules. These rules define how users are to interact with the system, and how the data is shared among the users. Moreover, once the rules are stored in the blockchain, they can not be altered without broadcasting to all nodes and verified by most of them.
%%% SETUP

\subsection{Setup}\label{sec:Setup}
Setup is the set of actions to configure the application. Setup phase is when a new supply chain is created or an existing one is updated or deleted. Users can also be created, updated and deleted by setup actions. When creating or editing a supply chain, Admin users will define which steps, sub-steps and information that the supply chain flow will contain, and which users from member group is allowed to add info and move asset to each step in the logistics network.

\subsubsection{Administrator}\label{sec:Administrator}
Administrator is the only user in Admin group. This actor has access to all areas of the program and have the same abilities that all other user types (configure, move asset and view flow), but his main responsibility is to configure the application, performing the setup actions. 


%%% DATA INSERTION

\subsection{Data Insertion}\label{sec:DataInsertion}

Data insertion are the actions that will fill the supply chain flow with data. Once a new supply chain is created by the admin user, it is ready to be populated with information. Member and admin users are responsible for perform these actions. In Data Insertion phase users can update information from a specific step and sub-step and move assets depending on the rules applied in the setup phase.

\subsubsection{Raw material/Producer}\label{sec:Rawmaterial}
Providers of raw materials, the quarrier or miner, a gravel producing company. Responsible for the extraction of raw materials (natural resources) and how they are firstly processed (pre-processed) by the suppliers or vendors in order to be delivered to the next stage.

\subsubsection{Manufacturer}\label{sec:Manufacturer}
The same as raw material/producer for gravel, or a machinery/equipment producer. Responsible for the process of converting the raw materials into products that are ready to sell.

\subsubsection{Distributor}\label{sec:Distributor}
Building materials shop or distributor. This actor is responsible of moving the output of the manufacturer (e.g., the product) from manufacturer’s site to a wholesaler and/or retailers.

\subsubsection{Wholesaler}\label{sec:Wholesaler}
Seller of gravel in large quantities, generally to businesses like building companies. Wholesalers in general do not sell products directly to the public, but to other retailers instead.

\subsubsection{Retailer}\label{sec:Retailer}
Seller of gravel in small quantities, generally to natural persons.


%%% VISUALIZATION

\subsection{Visualization}\label{sec:Visualization}

Visualization actions can be performed by any user in the system but its main purpose is to provide to the end user the capability of track the flow off an asset from point of origin to point of consumption.

\subsubsection{End User/ Consumer}\label{sec:EndUser}
The buyer of gravel, either a company or a natural person. Most areas of the program for this user will be unavailable. They will be able only to view the flow of an asset in the supply chain.
%\section{Steps and Substeps}\label{sec:stepsAndSubsteps}

%\subsection{Exploration}\label{sec:Exploration}
%Exploration is the starting point of a supply chain. Phase which encompasses Mineral Prospecting and Geological Survey.

%\subsubsection{Prospecting/ Surveying}\label{sec:Prospecting}
%Prospecting means the search of mineral occurrences/surveying means the research on a mineralised area, previously defined during the prospecting phase to be applied as a Mineral Right from the Mining Authority.

%\subsubsection{Mining right application}\label{sec:Mining}
%Through handing over documents to the Mining Authority (in Brazil it is Agência Nacional de Mineração [ANM]).

%\subsubsection{Geological Survey}\label{sec:GeologicalSurvey}
%Geological Survey is a set of activities carried out to determine the features on quality and quantity of ore found in a given mineral deposit.

%\subsubsection{Mineral reserve definition}\label{sec:MineralReserve}
%this step is related to the quantity of ore found during the Geological Survey.

%\subsection{Mine Development and Design}\label{sec:MineDevelopment}
%In this phase is made the preparation of a deposit to become a mine, opening accesses, setting up fences, building benches.


%\subsection{Production}\label{sec:Production}
%Production phase including ore extraction and mineral processing or transformation.

%\subsubsection{Quarrying/ Blasting plan}\label{sec:Quarrying}
%Quantity of stone extracted, same quantity of mineral reserves depleted, type of stone extracted.

%\subsubsection{Processing}\label{sec:Processing}
%Quantity of stone crushed/processed, same quantity of stone in stock deducted, type of stone product, market requirements.

%\subsubsection{Aggregate Product}\label{sec:AggregateProduct}
%Listing of products with identified quantities and respective technological features.

%\subsection{Distribution}\label{sec:Distribution}
%Product distribution with identification of clients and their segmentation. 

%\subsubsection{Client by Product Type}\label{sec:Client}
%Name, address, size, sector, business registration numbers (federal, state, municipal, county).

%\subsubsection{Transport}\label{sec:Transport}
%Invoice, bills, type of transportation, weight, distance.

%\subsection{Mine Closure}\label{sec:MineClosure}
%Mine Closure or mineral decommissioning, it means the end of the mining activity and the start of the installations removal and area remodelling through plantation and/or building a lake.

%\subsubsection{Legal Compliance/Decommissioning Plan}\label{sec:LegalCompliance}
%Status of all documentation related to the Decommissioning Plan and its Legal Compliance.

%\subsubsection{Environmental Recovery/ ANM %Application}\label{sec:EnvironmentalRecovery}
%Application number and status.
%\input{chapters/4_CONCLUSION/4_CONCLUSION.tex}

%% Parte pos-textual
\backmatter

% Bibliografia
\bibliographystyle{abntex2-alf}
\bibliography{biblio}

% Apendices
\appendix

%\lipsum
% Eh aconselhavel criar cada apendice em um arquivo separado, digamos
% "apendice1.tex", "apendice.tex", ... "apendiceM.tex" e depois
% inclui--los com:
% \include{apendice1}
% \include{apendice2}
% ...
% \include{apendiceM}

\xchapter{Project Management}{} %sem preambulo

\section{Activities}{} %sem preambulo

Once determined the software architecture and its main modules, these components were divided into activities for better project management. These activities are listed below, segregated by the main modules.

\subsection{Front end}\label{sec:FrontendActivities}
\begin{itemize}
\item Create login page
\item Create application configuration module
\item Create user creation module (actors)
\item Create data entry module (forms)
\item Create data visualization module
\item Create report module
\item Create authentication service
\item Create application setup service
\item Create user creation service (actors)
\item Create data entry service (forms)
\item Create data visualization service
\item Create reporting service
\end{itemize}

\subsection{Back end}\label{sec:BackendActivities}
\begin{itemize}
\item Gateway Creation
    \begin{itemize}
    \item Create authentication endpoint
    \item Create application configuration endpoint
    \item Create user creation endpoint (actors)
    \item Create data entry endpoint (forms)
    \item Create data visualization endpoint
    \item Create report endpoint
    \end{itemize}
\item Service Creation
    \begin{itemize}
    \item Create authentication service
    \item Create application setup service
    \item Create user creation service (actors)
    \item Create data entry service (forms)
    \item Create data visualization service
    \item Create reporting service
    \end{itemize}
\item Resource Locator Creation
    \begin{itemize}
    \item Create Connector with Filesystem
    \item Create Connector with Hyperledger Blockchain
    \item Create Connector with Oracle Database
    \end{itemize}
\end{itemize}

\subsection{Hyperledger Blockchain}\label{sec:HyperledgerBlockchain}
\begin{itemize}
\item Create chaincode (Smart contract)
\end{itemize}

\section{User stories}{} %sem preambulo

The table \ref{table:userStories} present all the user stories for artifacts development, used in the system and managed according to agile methodologies.

\begin{table}[H]
\caption{\ac{SCM-BP} User Stories}
\label{table:userStories}
    \begin{tabular}{|l|p{13.5cm}|}
    \hline 
    US-1  & As an administrator, clicking “new” on the supply chain list page takes you to the chain configuration page, with empty settings (no phases, subphases, and fields).\\
    \hline 
    US-2  & As an administrator, clicking “edit” on the supply chain list page takes you to the chain configuration page, with the settings filled in (with phases, subphases, and fields already registered). \\
    \hline
    US-3  & As an administrator, when you click delete on the supply chain list page, a modal should appear requesting deletion confirmation.\\
    \hline
    US-4  & As an administrator, when you click confirm deletion on the supply chain list page, an alert should appear stating the deletion result: alert-success or alert-danger.\\
    \hline
    US-5  & As an administrator, on the creation or editing screens of a chain, the administrator must tell from each section which user types can enter information in that section.\\
    \hline
    US-6  & As an administrator, clicking new on the User list page takes you to the user creation page with its empty settings.\\
    \hline
    US-7  & As an administrator, on the user creation page you have to enter the type of user (Admin, Producer, Manufacturer, Distributor, Wholesaler, Retailer, End User).\\
    \hline
    US-8  & As an administrator, clicking “edit” on the User list page takes you to the user creation page, with the settings filled in (with the previously entered data).\\
    \hline
    US-9  & As an administrator, when you click delete on the User list page, a modal should appear asking for deletion confirmation.\\
    \hline
    US-10 & As an administrator, when clicking confirm deletion on the User list page, an alert should appear stating the deletion result: alert-success or alert-danger.\\
    \hline
    US-11 & As "Member" User (Admin, Producer, Manufacturer, Distributor, Wholesaler, Retailer), clicking Move Asset takes you to the information entry page in the chain.\\
    \hline
    US-12 & As "Member", each user can only enter information regarding the allowed phase in the access rules (eg a distributor cannot enter exploration information) as defined in the use case 5.\\
    \hline
    US-13 & As any user (Admin, Member or End User), clicking Track Asset will take you to a page with a list of all assets paged and filtered by date in descending order (most current to oldest).\\
    \hline
    US-14 & As any user (Admin, Member or End User), by clicking on “Track Asset”, the user can enter in the input search an Id to search.\\
    \hline
    US-15 & As any user (Admin, Member or End User), by clicking on “Track”, the user will go to a page with all information of the respective asset, from its conception to the current state.\\
    \hline
    \end{tabular}
\end{table}
\section{Non-functional requirements}{} %sem preambulo

\begin{table}[H]
\caption{Non-functional requirements of \ac{SCM-BP}}
\label{table:rnf}
\begin{tabular}{|p{3cm}|p{12cm}|}
\hline
NF-1: Usability & The available product, corresponding APIs and documentation should be clear enough to allow for the developers to perform the implementation.\\
\hline
NF-2:  \newline Performance &  \textbf{Speed and latency}:  The throughput and latency on Hyperledger have already been tested, and the throughput is not expected to be as high as in a centralized data system. But, overall, the time to synchronize the information from one company to another might increase; The goal is to make the product be as fast as needed to support the businesses, even if it does not have better performance than other alternatives, since what the target here is the addition of new functionalities (shared ledger); \newline
\textbf{Precision and accuracy}: The product shall record the data just as it was entered, and predictions as to whether a product has any mismatching entries shall always be justifiable; \newline
\textbf{Reliability and availability}: The product shall not always be available unless all of the nodes fail at once, which is almost impossible, unless a coordinated attack were to happen; If some of the nodes happen to fail, the response time of the system might be lower than expected; \newline
\textbf{Scalability}: The product should scale to hundreds of companies, which would require a similar number of nodes;
\\
\hline
NF-3:  \newline Maintainability and portability & The product is expected to run on Linux based systems, compatible with the Docker, nodejs and golang versions that Hyperledger Fabric uses. More specifically, Oracle Cloud services have servers with the required setup for this. Creating new nodes or moving an existing one should be an easy process, without much complication, other than starting the node software on the environment, and closing an existing one, if needed. \\
\hline
NF-4: Security & 
\textbf{Privacy}: The system must ensure appropriate visibility of transactions and products, which might be privacy sensitive; sharing some data would pose a threat or could possibly have negative effects for some of the companies; otherwise, transactions should also be secure, authenticated and verifiable; \newline
\textbf{Immutability}: No one can make changes to the contents of the ledger;\newline
\textbf{Authorization}: All changes to any data should be approved by the people that possess the data or will be affected by these changes directly. A shipment delivery transaction should, for instance, be approved by both the person delivering and the person receiving the shipment. \\
\hline
\end{tabular}
\end{table}
\section{User creation fields}{} \label{app:userCreationFields}

\begin{table}[H]
\centering
\caption{User creation service fields}
\begin{tabular}{|p{3cm}|p{12cm}|}
\hline
Field & Input Value \\ 
\hline
User ID & Create a unique identifier for this user's login user name. Duplicated Id is not allowed. \\ 
\hline
First name & Enter the user's full first name. \\ 
\hline
Last name & Enter the user's last name. \\ 
\hline
Title & Enter a title or job description, or select one from the list. \\ 
\hline
Department & Select the user's department from the list. \\ 
\hline
Password & Assign a password to the user. This password can be permanent or temporary. \\ 
\hline
Password needs reset & Select this check box to require the user to change the password during the first login. \\ 
\hline
Locked out & Select this check box to lock the user out of the instance and terminate all their active sessions. \\ 
\hline
Active & Select this check box to make this user active. Only the administrator sees inactive user in: 
\begin{itemize}
\item Lists of users 
\item The selection list on reference fields (magnifying glass icon) 
\item The auto-complete list that appears when you type into a reference field.
\end{itemize} \\ 
\hline
Web service access only & Select this check box to designate this user as a non-interactive user. This field is available with Non-Interactive Sessions. \\ \hline
Internal Integration User & Select this check box to designate this user as an internal integration user. \\ 
\hline
Date format & Select the user's preferred format for dates. \\ 
\hline
Email & Enter the user's email address. \\ 
\hline
Notification & Select the type of notification to send to this user. The default is Email. \\ 
\hline
Time zone & Select the user's time zone. \\ 
\hline
Business phone & Enter this user's business phone number. \\ 
\hline
Mobile phone & Enter this user's mobile phone number. \\ 
\hline
Photo & Attach a photo of the user, if appropriate. \\ 
\hline
Geolocation tracked & Select the check box to enable location tracking. \\ 
\hline
Location & Select the user's usual location. This field is visible when geolocation is active. \\ 
\hline
\end{tabular}
\end{table}

\xchapter{Data Structure}{} %sem preambulo
%\xchapter{Data Structure}{} %sem preambulo

%\section{Data Structure}{} %sem preambulo

\begin{figure}[H]
\begin{center}
  \includegraphics[scale=0.95]{images/classDiagram.png}
\caption{\ac{SCM-BP} data structure}
\label{fig:dataStructure}
\end{center}
\end{figure}
%% Fim do documento
\end{document}
%------------------------------------------------------------------------------------------%
